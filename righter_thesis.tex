%% ============================================================
%%  Robert Righter – Master's Thesis [WIP]
%%  AMS Style, Option A Notation
%% ============================================================
\documentclass{amsart}

\usepackage{amsmath}
\usepackage{amsthm}
\usepackage{amssymb}
\usepackage{mathtools}
\usepackage[margin=1.25in]{geometry}
\allowdisplaybreaks

%% ------ Theorem-style Environments ------
\theoremstyle{definition}
\newtheorem{definition}{Definition}
\newtheorem{example}{Example}

\theoremstyle{plain}
\newtheorem{theorem}{Theorem}
\newtheorem{corollary}{Corollary}

%% ------ Title Block ------
\title[Recursive Computation of Compound Random Variables with a Finite-Mixture Count Distribution]{Recursive Computation of Compound Random Variables with a Finite-Mixture Count Distribution (Poisson, Binomial, and Negative Binomial Components)}
\author{Robert Righter}
\date{}

\begin{document}

\maketitle

%% ============================================================
\section{Introduction}
%% ============================================================

Compound random variables have been studied extensively in probability theory and applied
fields, as they provide a natural framework for modeling aggregate quantities subject to two
layers of randomness: the distribution of the summands and the distribution of the count
variable. Classical treatments can be found in
\cite{TODO},
where compound distributions are developed in the context of actuarial mathematics and
discrete distributions.

While the concept is straightforward, calculating the exact probability distribution of a
compound random variable can be a significant computational challenge.  Direct methods
involving multiple convolutions are often cumbersome and inefficient.  A more elegant and
powerful approach is the use of recursive methods, which allow for the efficient calculation
of the probability $P(S_N = k)$ based on the probabilities of preceding values.  This
thesis explores one such recursive technique, which is particularly effective when the
compounding distribution belongs to a specific class of distributions.

The primary contribution of this work is the extension and simplification of this recursive
method to the case where the compounding distribution is a finite mixture of Poisson,
Binomial and Negative Binomial distributions.  A direct application of the recursive formula
to this case leads to a computationally intensive, nested problem, where the mixing weights
of the distribution must be recalculated at each step.  Additionally, we derive a
closed-form expression that computes these recursive weights directly, thereby eliminating
some of the nested recursion and simplifying the calculation.

%% ============================================================
\section{Background}
%% ============================================================

\begin{definition}
The expected value of a discrete random variable, $X$, is denoted by $E(X)$, and given by
\[
    E[X] = \sum_{x=1}^{\infty} x \cdot f(x),
\]
where $f(x)$ is the probability mass function of $X$.
\end{definition}

\begin{definition}
Let $X_1, X_2, \ldots, X_n$ be a sequence of independent and identically distributed random
variables.  Let $N$ be a nonnegative integer valued random variable that is independent of
the sequence $X_1, X_2, \ldots, X_n$.  Then,
\[
    S_N = \sum_{i=1}^{N} X_i
\]
is a \emph{compound random variable}.
\end{definition}

\begin{definition}
In a compound random variable, the distribution of $N$ is called the \emph{compounding
distribution}.
\end{definition}

\begin{theorem}[Compound Random Variable Identity]
Let $M$ be a random variable that is independent of the sequence of random variables
$X_1, X_2, \ldots, X_n$ that forms a compound random variable and $M$ satisfies
\[
    P(M = n) = \frac{n\, P(N = n)}{E[N]}, \qquad n = 1, 2, 3, \ldots
\]
Then, for any function $h$,
\[
    E\!\left(S_N \cdot h(S_N)\right) = E[N] \cdot E\!\left(X_1 \cdot h(S_M)\right).
\]
\end{theorem}

\begin{proof}
\begin{align*}
E\!\left(S_N \cdot h(S_N)\right)
    &= E\!\left[\sum_{i=1}^{N} X_i\, h(S_N)\right] \\
    &= E\!\left[E\!\left(\sum_{i=1}^{N} X_i\, h(S_N) \;\Bigg|\; N = n\right)\right] \\
    &= \sum_{n=0}^{\infty} E\!\left(\sum_{i=1}^{n} X_i\, h(S_n) \;\Bigg|\; N = n\right)
       P(N = n) \\
    &= \sum_{n=0}^{\infty} E\!\left(\sum_{i=1}^{n} X_i\, h(S_n)\right) P(N = n) \\
    &= \sum_{n=0}^{\infty} \sum_{i=1}^{n} E\!\left(X_i\, h(S_n)\right) P(N = n).
\end{align*}
Since $X_1, X_2, \ldots$ are independent and identically distributed, we have
\[
    E\!\left(X_i \cdot h(S_N)\right) = E\!\left(X_j \cdot h(S_N)\right),
    \qquad \forall\, i, j \in \{1, 2, \ldots, n\}.
\]
So,
\begin{align*}
E\!\left(S_N \cdot h(S_N)\right)
    &= \sum_{n=0}^{\infty} \sum_{i=1}^{n} E\!\left(X_1\, h(S_n)\right) P(N = n) \\
    &= \sum_{n=1}^{\infty} n \cdot E\!\left(X_1\, h(S_n)\right) P(N = n).
\end{align*}
Then, by substitution of $P(M = n) = \dfrac{n\,P(N=n)}{E[N]}$, we have
\begin{align*}
    &= \sum_{n=0}^{\infty} E[N] \cdot E\!\left(X_1\, h(S_n)\right) P(M = n) \\
    &= E[N] \sum_{n=0}^{\infty} E\!\left(X_1\, h(S_n)\right) P(M = n) \\
    &= E[N] \sum_{n=0}^{\infty} E\!\left(X_1\, h(S_n) \mid M = n\right) P(M = n) \\
    &= E[N] \sum_{n=0}^{\infty} E\!\left(X_1\, h(S_M) \mid M = n\right) P(M = n) \\
    &= E[N]\, E\!\left(E\!\left(X_1\, h(S_M) \mid M = n\right)\right) \\
    &= E[N]\, E\!\left(X_1\, h(S_M)\right). \qedhere
\end{align*}
\end{proof}

\begin{corollary}\label{cor:main}
Let $S_N$ be a compound random variable and suppose $X_1, X_2, \ldots, X_n$ are positive
integer valued random variables.  Suppose $P(X_1 = i) = \alpha_i$, $i > 0$, and $M$ is a
random variable that is independent of the sequence $X_1, X_2, \ldots, X_n$ and satisfies
\[
    P(M = n) = \frac{n\,P(N = n)}{E[N]}.
\]
Then,
\[
    P(S_N = 0) = P(N = 0)
    \quad \text{and} \quad
    P(S_N = k) = \frac{1}{k}\, E[N] \sum_{i=1}^{k} i\,\alpha_i\, P(S_{M-1} = k - i).
\]
\end{corollary}

\begin{proof}
For a fixed $k$, let
\[
    h(x) = \begin{cases} 1 & \text{if } x = k, \\ 0 & \text{otherwise,} \end{cases}
\]
and note that
\[
    S_N \cdot h(S_N) = \begin{cases} k & \text{if } S_N = k, \\ 0 & \text{otherwise.} \end{cases}
\]
Therefore,
\[
    E\!\left(S_N \cdot h(S_N)\right) = k \cdot P(S_N = k).
\]
Now, by the Compound Random Variable Identity,
$E(S_N \cdot h(S_N)) = E[N] \cdot E(X_1 \cdot h(S_M))$.  So,
\begin{align*}
k \cdot P(S_N = k)
    &= E[N] \cdot E\!\left(X_1 \cdot h(S_M)\right) \\
    &= E[N] \cdot E\!\left(E\!\left(X_1 \cdot h(S_M) \mid X_1 = j\right)\right) \\
    &= E[N] \sum_{j=1}^{\infty} E\!\left(X_1 \cdot h(S_M) \mid X_1 = j\right)
       \cdot P(X_1 = j) \\
    &= E[N] \sum_{j=1}^{\infty} E\!\left(X_1 \cdot h(S_M) \mid X_1 = j\right)
       \cdot \alpha_j \\
    &= E[N] \sum_{j=1}^{\infty} E\!\left(j \cdot h(S_M) \mid X_1 = j\right) \cdot \alpha_j \\
    &= E[N] \sum_{j=1}^{\infty} j \cdot E\!\left(h(S_M) \mid X_1 = j\right) \cdot \alpha_j.
\end{align*}
Now,
\[
    E\!\left(h(S_M) \mid X_1 = j\right)
    = \sum_{S_M'} h(S_M')\, f_{S_M \mid X_1}(S_M' \mid j)
    = P(S_M = k \mid X_1 = j).
\]
And,
\begin{align*}
P(S_M = k \mid X_1 = j)
    &= P\!\left(\sum_{i=1}^{m} X_i = k \;\Bigg|\; X_1 = j\right) \\
    &= P\!\left(j + \sum_{i=2}^{m} X_i = k\right)
     = P\!\left(\sum_{i=2}^{m} X_i = k - j\right).
\end{align*}
Letting $\ell = i - 1$,
\[
    = P\!\left(\sum_{\ell=1}^{m-1} X_\ell = k - j\right) = P(S_{M-1} = k - j).
\]
So,
\[
    k \cdot P(S_N = k) = E[N] \sum_{j=1}^{\infty} j \cdot \alpha_j \cdot P(S_{M-1} = k - j).
\]
Note, when $j > k$, $P(S_{M-1} = k - j) = 0$.  So,
\[
    k \cdot P(S_N = k) = E[N] \sum_{j=1}^{k} j \cdot \alpha_j \cdot P(S_{M-1} = k - j),
\]
and therefore,
\[
    P(S_N = k) = \frac{1}{k} \cdot E[N] \sum_{j=1}^{k} j \cdot \alpha_j \cdot P(S_{M-1} = k - j).
\]
Notice, by definition, for $N \geq 1$,
\[
    P(S_N = 0) = P\!\left(\sum_{j=i}^{N} X_i = 0\right) = 0,
\]
so $P(S_N = 0) = P(N = 0)$.
\end{proof}

%% ============================================================
\section{Examples}
%% ============================================================

For each of the distributions below, let $S_N = \sum_{i=1}^{N} X_i$ be a compound
random variable where
\[
    P(X_1 = 1) = 0.05, \quad P(X_1 = 2) = 0.4, \quad P(X_1 = 3) = 0.1,
\]
\[
    P(X_1 = 4) = 0.25, \quad P(X_1 = 5) = 0.2.
\]
Find $P(S_N = 4)$.

%% ------ Example 1: Poisson ------
\begin{example}[N in the Poisson Distribution]\label{ex:poisson}
Let $N$ be a random variable of the Poisson distribution with parameter $\lambda = 3$.
So the probability mass function for the distribution of $N$ is
\[
    P^{(s)}(N = n) = \frac{e^{-\lambda}\,\lambda^n}{n!}, \qquad n = 0, 1, 2, \ldots
\]
We claim that $E[N^{(s)}] = \lambda$.

\begin{proof}
\begin{align*}
E[N^{(s)}]
    &= \sum_{k=0}^{\infty} k \cdot \frac{e^{-\lambda}\,\lambda^k}{k!}
     = 0 + \sum_{k=1}^{\infty} k \cdot \frac{e^{-\lambda}\,\lambda^k}{k!}
     = \sum_{k=1}^{\infty} \frac{e^{-\lambda}\,\lambda^k}{(k-1)!} \\
    &= e^{-\lambda} \cdot \lambda \cdot \sum_{k=1}^{\infty} \frac{\lambda^{k-1}}{(k-1)!}.
\end{align*}
Now let $j = k - 1$.  So when $k = 1$, $j = 0$, and as $k \to \infty$, $j \to \infty$.
So,
\[
    E[N^{(s)}] = \lambda \cdot e^{-\lambda} \cdot \sum_{j=0}^{\infty} \frac{\lambda^j}{j!}.
\]
We recognize $\sum_{j=0}^{\infty} \tfrac{\lambda^j}{j!}$ as the Taylor series for $e^\lambda$.
So,
\[
    E[N^{(s)}] = \lambda \cdot e^{-\lambda} \cdot e^{\lambda} = \lambda. \qedhere
\]
\end{proof}

Therefore, $E[N^{(s)}] = \lambda = 3$.  So we have,
\begin{align*}
P(M^{(s)} - 1 = n)
    &= P(M^{(s)} = n + 1)
     = \frac{(n+1)\,P^{(s)}(N = n+1)}{E[N^{(s)}]} \\
    &= \frac{n+1}{\lambda} \cdot \frac{e^{-\lambda}\,\lambda^{n+1}}{(n+1)!}
     = \frac{e^{-\lambda}\,\lambda^n}{n!}.
\end{align*}
Therefore, we see that $P^{(s+1)}(N = n)$ is also the Poisson distribution with parameter
$\lambda$.  Since the distribution is unchanged at each recursion level, we have
\[
    P^{(s)}(S_N = k) = \frac{\lambda}{k} \sum_{i=1}^{k} i\,\alpha_i \cdot P^{(s+1)}(S_N = k - i),
\]
which, because $P^{(s)} = P^{(s+1)}$ for all $s \geq 0$, simplifies to
\[
    P(S_N = k) = \frac{\lambda}{k} \sum_{i=1}^{k} i\,\alpha_i \cdot P(S_N = k - i).
\]
Now we calculate $P(S_N = 4)$.
\begin{align*}
P(S_N = 0)
    &= P(N = 0) = \frac{e^{-\lambda}\,\lambda^0}{0!} = e^{-\lambda} = e^{-3}. \\
P(S_N = 1)
    &= \frac{3}{1}\bigl[1 \cdot (0.05) \cdot P(S_N = 0)\bigr]
     = 3 \cdot (0.05) \cdot e^{-3}
     = 0.15\,e^{-3}. \\
P(S_N = 2)
    &= \frac{3}{2}\bigl[(0.05) \cdot P(S_N = 1) + 2 \cdot (0.4) \cdot P(S_N = 0)\bigr] \\
    &= \frac{3}{2}\bigl[(0.05)(0.15\,e^{-3}) + (0.8)\,e^{-3}\bigr]
     = 1.21125\,e^{-3}. \\
P(S_N = 3)
    &= \frac{3}{3}\bigl[(0.05) \cdot P(S_N = 2) + (0.8) \cdot P(S_N = 1)
       + 3 \cdot (0.1) \cdot P(S_N = 0)\bigr] \\
    &= (0.05)(1.21125\,e^{-3}) + (0.8)(0.15\,e^{-3}) + (0.3)\,e^{-3}
     = 0.4805625\,e^{-3}. \\
P(S_N = 4)
    &= \frac{3}{4}\bigl[0.05 \cdot P(S_N = 3) + 0.8 \cdot P(S_N = 2)
       + 0.3 \cdot P(S_N = 1) + 4 \cdot (0.25) \cdot P(S_N = 0)\bigr] \\
    &= \frac{3}{4}\bigl[0.05(0.4805625\,e^{-3}) + 0.8(1.21125\,e^{-3})
       + 0.3(0.15\,e^{-3}) + e^{-3}\bigr] \\
    &= 1.528521094\,e^{-3}.
\end{align*}
\end{example}

%% ------ Example 2: Negative Binomial ------
\begin{example}[N in the Negative Binomial Distribution]\label{ex:negbin}
Let $N$ be a random variable of the negative binomial distribution with parameters
$r = 6$ and $p = 0.6$.  The probability mass function for the distribution of $N^{(s)}$ at
recursion level $s$ is
\[
    P^{(s)}(N = n) = \binom{n + r + s - 1}{n} p^{r+s}(1-p)^n, \qquad n \geq 0.
\]
We claim that $E[N^{(s)}] = \dfrac{(r+s)(1-p)}{p}$.

\begin{proof}
\begin{align*}
E[N^{(s)}]
    &= \sum_{k=0}^{\infty} k \binom{k + r + s - 1}{k} p^{r+s}(1-p)^k
     = \sum_{k=1}^{\infty} k \binom{k + r + s - 1}{k} p^{r+s}(1-p)^k.
\end{align*}
Recalling that $\dbinom{a}{b} = \dfrac{a!}{b!\,(a-b)!}$, we see that
\[
    k \binom{k + r + s - 1}{k}
    = k \cdot \frac{(k + r + s - 1)!}{k!\,(r + s - 1)!}
    = \frac{(k + r + s - 1)!}{(k-1)!\,(r + s - 1)!}
    = (r+s)\binom{k + r + s - 1}{k - 1}.
\]
So we have,
\[
    E[N^{(s)}]
    = (r+s)(1-p)\,p^{r+s} \sum_{k=1}^{\infty}
      \binom{k + r + s - 1}{k - 1}(1-p)^{k-1}.
\]
Let $m = k - 1$.  By substitution,
\[
    E[N^{(s)}] = (r+s)(1-p)\,p^{r+s}
    \sum_{m=0}^{\infty} \binom{m + (r+s)}{m}(1-p)^m.
\]
Recalling the negative binomial theorem,
$(1+x)^{-t} = \sum_{i=0}^{\infty}\binom{-t}{i} x^i$,
and applying it, we see that
\[
    \sum_{m=0}^{\infty} \binom{m + r + s}{m}(1-p)^m = p^{-(r+s+1)}.
\]
So by substitution,
\[
    E[N^{(s)}]
    = (r+s)(1-p)\,p^{r+s} \cdot p^{-(r+s+1)}
    = \frac{(r+s)(1-p)}{p}. \qedhere
\]
\end{proof}

So we have,
\begin{align*}
P^{(s)}(M^{(s)} - 1 = n)
    &= P^{(s)}(M^{(s)} = n + 1) \\
    &= \frac{(n+1)\,P^{(s)}(N = n+1)}{E[N^{(s)}]} \\
    &= \binom{n + r + s}{n + 1} p^{r+s}(1-p)^{n+1}
       \cdot \frac{(n+1)}{(r+s)(1-p)/p} \\
    &= \frac{(n+1)\,p}{(r+s)(1-p)} \cdot \frac{(n + r + s)!}{(n+1)!\,(r+s-1)!}
       \cdot p^{r+s}(1-p)^{n+1} \\
    &= \frac{(n + r + s)!}{n!\,(r+s)!} \cdot p^{r+s+1}(1-p)^n \\
    &= \binom{n + (r+s+1) - 1}{n} p^{r+s+1}(1-p)^n
     = P^{(s+1)}(N = n).
\end{align*}
Therefore, we see that $P^{(s)}(M^{(s)} - 1 = n)$ is also the negative binomial
distribution but at the next recursion level $(s+1)$.  So, we have
\[
    P^{(s)}(S_N = k)
    = \frac{(r+s)(1-p)}{k\,p} \sum_{i=1}^{k} i\,\alpha_i \cdot P^{(s+1)}(S_N = k - i).
\]
Now we can begin the computation.  With $r = 6$, $p = 0.6$ (so $1-p = 0.4$), at level
$s = 0$ the coefficient is $\dfrac{r(1-p)}{p} = 4$, giving
\[
    P^{(0)}(S_N = k) = \frac{4}{k} \sum_{i=1}^{k} i\,\alpha_i \cdot P^{(1)}(S_N = k - i).
\]

\medskip
\noindent\textit{Base cases} $P^{(s)}(S_N = 0) = p^{r+s}$:
\begin{align*}
P^{(0)}(S_N = 0) &= (0.6)^{6} = 0.046656. \\
P^{(1)}(S_N = 0) &= (0.6)^{7} = 0.0279936. \\
P^{(2)}(S_N = 0) &= (0.6)^{8} = 0.01679616. \\
P^{(3)}(S_N = 0) &= (0.6)^{9} = 0.010077696. \\
P^{(4)}(S_N = 0) &= (0.6)^{10} = 0.0060466176.
\end{align*}

\medskip
\noindent\textit{Computing} $P^{(s)}(S_N = 1)$:
\begin{align*}
P^{(0)}(S_N = 1)
    &= \frac{(r)(1-p)}{1 \cdot p}\bigl[0.05 \cdot P^{(1)}(S_N = 0)\bigr]
     = (0.2)(0.0279936)
     = 0.00559872. \\
P^{(1)}(S_N = 1)
    &= \frac{(r+1)(1-p)}{1 \cdot p}\bigl[0.05 \cdot P^{(2)}(S_N = 0)\bigr]
     = \tfrac{7 \cdot 0.4}{0.6}(0.05)(0.01679616)
     = 0.003919104. \\
P^{(2)}(S_N = 1)
    &= \frac{(r+2)(1-p)}{1 \cdot p}\bigl[0.05 \cdot P^{(3)}(S_N = 0)\bigr]
     = \tfrac{8 \cdot 0.4}{0.6}(0.05)(0.010077696)
     = 0.0026873856. \\
P^{(3)}(S_N = 1)
    &= \frac{(r+3)(1-p)}{1 \cdot p}\bigl[0.05 \cdot P^{(4)}(S_N = 0)\bigr]
     = \tfrac{9 \cdot 0.4}{0.6}(0.05)(0.0060466176)
     = 0.00181398528.
\end{align*}

\medskip
\noindent\textit{Computing} $P^{(s)}(S_N = 2)$:
\begin{align*}
P^{(1)}(S_N = 2)
    &= \frac{(r+1)(1-p)}{2\,p}
       \bigl[(0.05)\cdot P^{(2)}(S_N = 1) + (0.8)\cdot P^{(2)}(S_N = 0)\bigr] \\
    &= \frac{7 \cdot 0.4}{2 \cdot 0.6}
       \bigl[(0.05)(0.0026873856) + (0.8)(0.01679616)\bigr]
     = 0.03166636032. \\
P^{(2)}(S_N = 2)
    &= \frac{(r+2)(1-p)}{2\,p}
       \bigl[(0.05)\cdot P^{(3)}(S_N = 1) + (0.8)\cdot P^{(3)}(S_N = 0)\bigr] \\
    &= \frac{8 \cdot 0.4}{2 \cdot 0.6}
       \bigl[(0.05)(0.00181398528) + (0.8)(0.010077696)\bigr]
     = 0.021740949504.
\end{align*}

\medskip
\noindent\textit{Computing} $P^{(1)}(S_N = 3)$:
\begin{align*}
P^{(1)}(S_N = 3)
    &= \frac{(r+1)(1-p)}{3\,p}
       \bigl[(0.05)\cdot P^{(2)}(S_N = 2) + (0.8)\cdot P^{(2)}(S_N = 1)
       + (0.3)\cdot P^{(2)}(S_N = 0)\bigr] \\
    &= \frac{7 \cdot 0.4}{3 \cdot 0.6}
       \bigl[(0.05)(0.021740949504) + (0.8)(0.0026873856)
       + (0.3)(0.01679616)\bigr] \\
    &= 0.01287347282.
\end{align*}

\medskip
\noindent\textit{Computing} $P^{(0)}(S_N = 4)$:
\begin{align*}
P^{(0)}(S_N = 4)
    &= \frac{r(1-p)}{4\,p}
       \bigl[0.05 \cdot P^{(1)}(S_N = 3) + 0.8 \cdot P^{(1)}(S_N = 2)
       + 0.3 \cdot P^{(1)}(S_N = 1) + P^{(1)}(S_N = 0)\bigr] \\
    &= \frac{6 \cdot 0.4}{4 \cdot 0.6}
       \bigl[(0.05)(0.01287347282) + (0.8)(0.03166636032) \\
    &\qquad\qquad\quad + (0.3)(0.003919104) + 0.0279936\bigr] \\
    &= \boxed{0.05514609310}.
\end{align*}
\end{example}

%% ------ Example 3: Binomial ------
\begin{example}[N in the Binomial Distribution]\label{ex:binom}
Let $N$ be a random variable of the Binomial distribution with parameters $r = 6$ and
$p = 0.6$.  The probability mass function for the distribution of $N^{(s)}$ at recursion
level $s$ is
\[
    P^{(s)}(N = n) = \binom{r - s}{n} p^n (1-p)^{(r-s)-n},
    \qquad n \geq 0.
\]
We claim that $E[N^{(s)}] = (r - s)\,p$.

\begin{proof}
\begin{align*}
E[N^{(s)}]
    &= \sum_{k=0}^{\infty} k \binom{r-s}{k} p^k (1-p)^{(r-s)-k}
     = \sum_{k=1}^{\infty} k \binom{r-s}{k} p^k (1-p)^{(r-s)-k}.
\end{align*}
Notice that
\[
    k \binom{r-s}{k}
    = k \cdot \frac{(r-s)!}{k!\,(r-s-k)!}
    = \frac{(r-s)!}{(k-1)!\,(r-s-k)!}
    = (r-s)\binom{r-s-1}{k-1}.
\]
By substitution,
\[
    E[N^{(s)}]
    = (r-s)\,p \sum_{k=1}^{\infty} \binom{r-s-1}{k-1} p^{k-1}(1-p)^{(r-s)-k}.
\]
We let $m = k - 1$.  So we have,
\[
    E[N^{(s)}]
    = (r-s)\,p \sum_{m=0}^{\infty}
      \binom{r-s-1}{m} p^m (1-p)^{(r-s-1)-m}.
\]
And by the binomial theorem,
\[
    \sum_{m=0}^{\infty} \binom{r-s-1}{m} p^m (1-p)^{(r-s-1)-m}
    = \bigl[p + (1-p)\bigr]^{r-s-1} = 1.
\]
By substitution we have $E[N^{(s)}] = (r-s)\,p$. \qedhere
\end{proof}

So, for this example at $s = 0$, $E[N^{(0)}] = r\,p = 3.6$.  We have,
\begin{align*}
P^{(s)}(M^{(s)} - 1 = n)
    &= P^{(s)}(M^{(s)} = n + 1)
     = \frac{(n+1)\,P^{(s)}(N = n + 1)}{E[N^{(s)}]} \\
    &= \binom{r-s}{n+1} p^{n+1}(1-p)^{(r-s)-n-1}
       \cdot \frac{n+1}{(r-s)\,p} \\
    &= \frac{(r-s)!}{(n+1)!\,(r-s-n-1)!}\,p(1-p)^{(r-s-1)-n}
       \cdot \frac{n+1}{(r-s)\,p} \\
    &= \frac{(r-s-1)!}{n!\,(r-s-1-n)!} \cdot p^n (1-p)^{(r-s-1)-n} \\
    &= \binom{r-s-1}{n} p^n (1-p)^{(r-s-1)-n}
     = P^{(s+1)}(N = n).
\end{align*}
Therefore, we see that $P^{(s)}(M^{(s)} - 1 = n)$ is also the binomial distribution but at
the next recursion level $(s + 1)$, which corresponds to a parameter change to $(r - s - 1)$.
So, we have
\[
    P^{(s)}(S_N = k)
    = \frac{(r-s)\,p}{k} \sum_{i=1}^{k} i\,\alpha_i \cdot P^{(s+1)}(S_N = k - i).
\]
Now we can begin the computation to find $P(S_N = 4)$.

\medskip
\noindent\textit{Base cases} $P^{(s)}(S_N = 0) = (1-p)^{r-s}$:
\begin{align*}
P^{(1)}(S_N = 0) &= P^{(1)}(N = 0) = (0.4)^5 = 0.01024. \\
P^{(2)}(S_N = 0) &= P^{(2)}(N = 0) = (0.4)^4 = 0.0256. \\
P^{(3)}(S_N = 0) &= P^{(3)}(N = 0) = (0.4)^3 = 0.064.
\end{align*}

\medskip
\noindent
From Corollary~\ref{cor:main}, $P^{(0)}(S_N = 0) = P^{(0)}(N = 0) = (0.4)^6 = 0.004096$.

\begin{align*}
P^{(0)}(S_N = 1)
    &= \frac{r\,p}{1}\bigl[1 \cdot (0.05) \cdot P^{(1)}(S_N = 0)\bigr]
     = 6(0.6)(0.05)(0.01024) = 0.0018432. \\
P^{(3)}(S_N = 1)
    &= \frac{(r-3)\,p}{1}\bigl[0.05 \cdot P^{(4)}(S_N = 0)\bigr]
     = 3(0.6)(0.05)(0.4)^2 = 0.0144. \\
P^{(2)}(S_N = 1)
    &= \frac{(r-2)\,p}{1}\bigl[0.05 \cdot P^{(3)}(S_N = 0)\bigr]
     = 4(0.6)(0.05)(0.064) = 0.00768. \\
P^{(1)}(S_N = 1)
    &= \frac{(r-1)\,p}{1}\bigl[0.05 \cdot P^{(2)}(S_N = 0)\bigr]
     = 5(0.6)(0.05)(0.0256) = 0.00384.
\end{align*}

\begin{align*}
P^{(2)}(S_N = 2)
    &= \frac{(r-2)\,p}{2}
       \bigl[0.05 \cdot P^{(3)}(S_N = 1) + 0.8 \cdot P^{(3)}(S_N = 0)\bigr] \\
    &= \frac{4(0.6)}{2}\bigl[0.05(0.0144) + 0.8(0.064)\bigr]
     = 0.062304. \\
P^{(1)}(S_N = 2)
    &= \frac{(r-1)\,p}{2}
       \bigl[0.05 \cdot P^{(2)}(S_N = 1) + 0.8 \cdot P^{(2)}(S_N = 0)\bigr] \\
    &= \frac{5(0.6)}{2}\bigl[0.05(0.00768) + 0.8(0.0256)\bigr]
     = 0.031296.
\end{align*}

\begin{align*}
P^{(1)}(S_N = 3)
    &= \frac{(r-1)\,p}{3}
       \bigl[(0.05)\cdot P^{(2)}(S_N = 2) + (0.8)\cdot P^{(2)}(S_N = 1)
       + (0.3)\cdot P^{(2)}(S_N = 0)\bigr] \\
    &= \frac{5(0.6)}{3}
       \bigl[(0.05)(0.062304) + (0.8)(0.00768) + (0.3)(0.0256)\bigr]
     = 0.016937200.
\end{align*}

\begin{align*}
P^{(0)}(S_N = 4)
    &= \frac{r\,p}{4}
       \bigl[0.05 \cdot P^{(1)}(S_N = 3) + 0.8 \cdot P^{(1)}(S_N = 2)
       + 0.3 \cdot P^{(1)}(S_N = 1) + P^{(1)}(S_N = 0)\bigr] \\
    &= \frac{6(0.6)}{4}
       \bigl[(0.05)(0.016937200) + (0.8)(0.031296)
       + (0.3)(0.00384) + 0.01024\bigr] \\
    &= \boxed{0.033548184}.
\end{align*}
\end{example}

%% ============================================================
\section{Computation with a Finite Mixture of Poisson Distributions}
%% ============================================================

We will now consider the computation of compound random variable probabilities when the
counting distribution is a mixture of Poisson distributions.

\begin{example}[N in a Finite Mixture of Two Poisson Distributions]\label{ex:poisson-mix}
Consider a discrete mixture distribution where the event is generated by one of two random
processes.  The event is either generated by process one, which follows a Poisson
distribution with mean $\lambda_1$, or process two, which follows a Poisson distribution
with mean $\lambda_2$.  The probability of process one is $\beta_1$, and the probability of
process two is $\beta_2$, where $\beta_1 + \beta_2 = 1$.  Let the parameters of this
distribution be $\lambda_1, \lambda_2, \beta_1, \beta_2$.

So, we have,
\[
    P^{(s)}(N = n)
    = \beta_1^{(s)} \cdot \frac{e^{-\lambda_1}\,\lambda_1^n}{n!}
    + \beta_2^{(s)} \cdot \frac{e^{-\lambda_2}\,\lambda_2^n}{n!},
    \qquad n = 0, 1, 2, \ldots
\]
We claim that $E[N^{(s)}] = \beta_1^{(s)}\lambda_1 + \beta_2^{(s)}\lambda_2$.

\begin{proof}
\begin{align*}
E[N^{(s)}]
    &= \sum_{k=0}^{\infty} k \left[\beta_1^{(s)} \cdot
       \frac{e^{-\lambda_1}\,\lambda_1^k}{k!}
       + \beta_2^{(s)} \cdot \frac{e^{-\lambda_2}\,\lambda_2^k}{k!}\right] \\
    &= \beta_1^{(s)} \sum_{k=0}^{\infty} k \cdot \frac{e^{-\lambda_1}\,\lambda_1^k}{k!}
     + \beta_2^{(s)} \sum_{k=0}^{\infty} k \cdot \frac{e^{-\lambda_2}\,\lambda_2^k}{k!}.
\end{align*}
Recall from Example~\ref{ex:poisson} that we showed
$\sum_{k=0}^{\infty} k \cdot \dfrac{e^{-\lambda_i}\,\lambda_i^k}{k!} = \lambda_i$.
So, by substitution we have $E[N^{(s)}] = \beta_1^{(s)}\lambda_1 + \beta_2^{(s)}\lambda_2$.
\qedhere
\end{proof}

So, we have,
\begin{align*}
P(M^{(s)} - 1 = n)
    &= P(M^{(s)} = n + 1)
     = \frac{(n+1)\,P^{(s)}(N = n+1)}{E[N^{(s)}]} \\
    &= \frac{(n+1)\!\left[\beta_1^{(s)} \cdot \dfrac{e^{-\lambda_1}\lambda_1^{n+1}}{(n+1)!}
       + \beta_2^{(s)} \cdot \dfrac{e^{-\lambda_2}\lambda_2^{n+1}}{(n+1)!}\right]}
      {\beta_1^{(s)}\lambda_1 + \beta_2^{(s)}\lambda_2} \\
    &= \frac{\beta_1^{(s)}\lambda_1 \cdot \dfrac{e^{-\lambda_1}\lambda_1^n}{n!}
       + \beta_2^{(s)}\lambda_2 \cdot \dfrac{e^{-\lambda_2}\lambda_2^n}{n!}}
      {\beta_1^{(s)}\lambda_1 + \beta_2^{(s)}\lambda_2} \\
    &= \frac{\beta_1^{(s)}\lambda_1}{\beta_1^{(s)}\lambda_1 + \beta_2^{(s)}\lambda_2}
       \cdot \frac{e^{-\lambda_1}\lambda_1^n}{n!}
     + \frac{\beta_2^{(s)}\lambda_2}{\beta_1^{(s)}\lambda_1 + \beta_2^{(s)}\lambda_2}
       \cdot \frac{e^{-\lambda_2}\lambda_2^n}{n!}.
\end{align*}
Notice, this is also a finite mixture distribution where both processes follow the Poisson
distribution with $\lambda_1, \lambda_2$ respectively.  However, we have updated values for
$\beta_1^{(s)}$ and $\beta_2^{(s)}$.  The weight update rule is
\[
    \beta_j^{(s+1)}
    = \frac{\beta_j^{(s)}\,\lambda_j}{\beta_1^{(s)}\lambda_1 + \beta_2^{(s)}\lambda_2}.
\]
So, we have
\[
    P^{(s)}(S_N = k)
    = \frac{E[N^{(s)}]}{k} \sum_{i=1}^{k} i\,\alpha_i \cdot P^{(s+1)}(S_N = k - i).
\]
Now we can begin the computation.  Let $\lambda_1 = 3$, $\lambda_2 = 4$,
$\beta_1^{(0)} = 0.6$, $\beta_2^{(0)} = 0.4$.  We want to find $P^{(0)}(S_N = 4)$.

\medskip
\noindent\textit{Computing the mixing weights at each level}:
\begin{align*}
\beta_1^{(0)} &= 0.6, \quad \beta_2^{(0)} = 0.4, \quad
E[N^{(0)}] = (0.6)(3) + (0.4)(4) = 3.4. \\
\beta_1^{(1)} &= \frac{(0.6)(3)}{3.4} = \frac{1.8}{3.4}, \quad
\beta_2^{(1)} = \frac{(0.4)(4)}{3.4} = \frac{1.6}{3.4}, \quad
E[N^{(1)}] = \frac{1.8(3) + 1.6(4)}{3.4} = \frac{11.8}{3.4}. \\
\beta_1^{(2)} &= \frac{5.4}{11.8}, \quad \beta_2^{(2)} = \frac{6.4}{11.8}, \quad
E[N^{(2)}] = \frac{5.4(3) + 6.4(4)}{11.8} = \frac{41.8}{11.8}. \\
\beta_1^{(3)} &= \frac{16.2}{41.8}, \quad \beta_2^{(3)} = \frac{25.6}{41.8}, \quad
E[N^{(3)}] = \frac{16.2(3) + 25.6(4)}{41.8} = \frac{151}{41.8}. \\
\beta_1^{(4)} &= \frac{48.6}{151}, \quad \beta_2^{(4)} = \frac{102.4}{151}.
\end{align*}

\medskip
\noindent\textit{Base cases} $P^{(s)}(S_N = 0) = \beta_1^{(s)}e^{-\lambda_1}
+ \beta_2^{(s)}e^{-\lambda_2}$:
\begin{align*}
P^{(1)}(S_N = 0) &= \tfrac{1.8}{3.4}\,e^{-3} + \tfrac{1.6}{3.4}\,e^{-4}. \\
P^{(2)}(S_N = 0) &= \tfrac{5.4}{11.8}\,e^{-3} + \tfrac{6.4}{11.8}\,e^{-4}. \\
P^{(3)}(S_N = 0) &= \tfrac{16.2}{41.8}\,e^{-3} + \tfrac{25.6}{41.8}\,e^{-4}. \\
P^{(4)}(S_N = 0) &= \tfrac{48.6}{151}\,e^{-3} + \tfrac{102.4}{151}\,e^{-4}.
\end{align*}

\medskip
\noindent\textit{Computing} $P^{(s)}(S_N = 1)$:
\begin{align*}
P^{(1)}(S_N = 1)
    &= \frac{E[N^{(1)}]}{1}\bigl[1 \cdot 0.05 \cdot P^{(2)}(S_N = 0)\bigr] \\
    &= \frac{11.8}{3.4}\left[(0.05)\!\left(\frac{5.4}{11.8}\,e^{-3}
       + \frac{6.4}{11.8}\,e^{-4}\right)\right] \\
    &= 0.07941176471\,e^{-3} + 0.09411764706\,e^{-4}. \\
P^{(2)}(S_N = 1)
    &= \frac{E[N^{(2)}]}{1}\bigl[0.05 \cdot P^{(3)}(S_N = 0)\bigr] \\
    &= 0.06864406780\,e^{-3} + 0.10847457627\,e^{-4}. \\
P^{(3)}(S_N = 1)
    &= \frac{E[N^{(3)}]}{1}\bigl[0.05 \cdot P^{(4)}(S_N = 0)\bigr] \\
    &= 0.05813397129\,e^{-3} + 0.12248803828\,e^{-4}.
\end{align*}

\medskip
\noindent\textit{Computing} $P^{(s)}(S_N = 2)$:
\begin{align*}
P^{(1)}(S_N = 2)
    &= \frac{E[N^{(1)}]}{2}
       \bigl[(0.05)\cdot P^{(2)}(S_N = 1) + (0.8)\cdot P^{(2)}(S_N = 0)\bigr] \\
    &= 0.64125\,e^{-3} + 0.76235294118\,e^{-4}. \\
P^{(2)}(S_N = 2)
    &= \frac{E[N^{(2)}]}{2}
       \bigl[(0.05)\cdot P^{(3)}(S_N = 1) + (0.8)\cdot P^{(3)}(S_N = 0)\bigr] \\
    &= 0.55430084746\,e^{-3} + 0.87864406780\,e^{-4}.
\end{align*}

\medskip
\noindent\textit{Computing} $P^{(1)}(S_N = 3)$:
\begin{align*}
P^{(1)}(S_N = 3)
    &= \frac{E[N^{(1)}]}{3}
       \bigl[(0.05)\cdot P^{(2)}(S_N = 2)
       + (0.8)\cdot P^{(2)}(S_N = 1)
       + (0.3)\cdot P^{(2)}(S_N = 0)\bigr] \\
    &= 0.25441544118\,e^{-3} + 0.33945098039\,e^{-4}.
\end{align*}

\medskip
\noindent\textit{Finally,} $P^{(0)}(S_N = 4)$:
\begin{align*}
P^{(0)}(S_N = 4)
    &= \frac{E[N^{(0)}]}{4}
       \bigl[(0.05)\cdot P^{(1)}(S_N = 3) + (0.8)\cdot P^{(1)}(S_N = 2) \\
    &\qquad\qquad\qquad + (0.3)\cdot P^{(1)}(S_N = 1)
       + P^{(1)}(S_N = 0)\bigr] \\
    &= \frac{3.4}{4}
       \bigl[(0.05)(0.25441544118\,e^{-3} + 0.33945098039\,e^{-4}) \\
    &\qquad\qquad + (0.8)(0.64125\,e^{-3} + 0.76235294118\,e^{-4}) \\
    &\qquad\qquad + (0.3)(0.07941176471\,e^{-3} + 0.09411764706\,e^{-4}) \\
    &\qquad\qquad + \bigl(\tfrac{1.8}{3.4}\,e^{-3} + \tfrac{1.6}{3.4}\,e^{-4}\bigr)\bigr] \\
    &= \boxed{0.91711265625\,e^{-3} + 0.95682666667\,e^{-4}}.
\end{align*}
\end{example}

%% ============================================================
\section{Finite Poisson Mixture Generalization}
%% ============================================================

Next, we will describe a more general case of the Finite Mixture of Poisson Distributions.
Consider a discrete mixture distribution where the event is generated by one or more random
processes.  The event is generated by one of $t$ processes, which follows a Poisson
distribution with mean $\lambda_i$ for $i \in \mathbb{Z}$ with $0 < i \leq t$.  The
probability of process $i$ is $\beta_i$ for $i \in \mathbb{Z}$ with $0 < i \leq t$ and
where $\beta_1 + \beta_2 + \cdots + \beta_t = 1$.

Now we are ready to define our probability mass function.  Let $s \in \mathbb{Z}$ with
$0 \leq s$, and let $s$ represent the level of recursion.  Let the parameters of this
distribution be $s, \lambda_1, \ldots, \lambda_t$.
\[
    P^{(s)}(N = n)
    = \beta_1^{(s)} \cdot \frac{e^{-\lambda_1}\lambda_1^n}{n!}
    + \beta_2^{(s)} \cdot \frac{e^{-\lambda_2}\lambda_2^n}{n!}
    + \cdots
    + \beta_t^{(s)} \cdot \frac{e^{-\lambda_t}\lambda_t^n}{n!},
    \qquad n \geq 0.
\]

\begin{corollary}[Recursive Properties of the Generalized Finite Mixture of Poisson
Distributions]\label{cor:poisson-gen}
Let $N^{(s)}$ be a mixed Poisson random variable at recursion level $s$ with probability
mass function
\[
    P^{(s)}(N = n)
    = \sum_{j=1}^{t}
      \left(\frac{\beta_j\,\lambda_j^s}{\sum_{i=1}^{t}\beta_i\,\lambda_i^s}\right)
      \cdot \frac{e^{-\lambda_j}\lambda_j^n}{n!},
\]
where $\sum_{i=1}^{t}\beta_i = 1$, $\beta_i > 0$, and $\lambda_i > 0$.  Let $M^{(s)}$ be
the random variable satisfying the compound identity condition associated with $N^{(s)}$.
We claim that for all $s \geq 0$:
\begin{enumerate}
\item $\displaystyle E[N^{(s)}]
      = \frac{\sum_{i=1}^{t}\beta_i\,\lambda_i^{s+1}}
             {\sum_{j=1}^{t}\beta_j\,\lambda_j^s}.$
\item The distribution of $M^{(s)} - 1$ is identical to the distribution of $N^{(s+1)}$.
\end{enumerate}
\end{corollary}

\begin{proof}[Proof (by induction)]
\textit{Base Case} $(s = 0)$: For $s = 0$, the weights are $\beta_i$.  First, we compute
the expected value of $E[N^{(0)}]$:
\begin{align*}
E[N^{(0)}]
    &= \sum_{n=0}^{\infty} n \left[\sum_{i=1}^{t}\beta_i \cdot
       \frac{e^{-\lambda_i}\lambda_i^n}{n!}\right]
     = \sum_{i=1}^{t}\beta_i\,\lambda_i.
\end{align*}
This matches the form of Claim~1 where the denominator
$\sum_{i=1}^{t}\beta_i\,\lambda_i^0 = \sum_{i=1}^{t}\beta_i = 1$.

Next, we apply the compound random identity and see that,
\begin{align*}
P(M^{(0)} - 1 = n)
    &= P(M^{(0)} = n + 1)
     = \frac{(n+1)\,P^{(0)}(N = n+1)}{E[N^{(0)}]} \\
    &= \frac{(n+1)\!\left[\beta_1 \cdot
       \dfrac{e^{-\lambda_1}\lambda_1^{n+1}}{(n+1)!}
       + \cdots + \beta_t \cdot
       \dfrac{e^{-\lambda_t}\lambda_t^{n+1}}{(n+1)!}\right]}
      {\beta_1\lambda_1 + \cdots + \beta_t\lambda_t} \\
    &= \frac{n+1}{\sum_{j=1}^{t}\beta_j\lambda_j}
       \cdot \sum_{i=1}^{t}\beta_i \cdot
       \frac{e^{-\lambda_i}\lambda_i^{n+1}}{(n+1)!} \\
    &= \sum_{i=1}^{t}
       \frac{\beta_i\,\lambda_i}{\sum_{j=1}^{t}\beta_j\,\lambda_j}
       \cdot \frac{e^{-\lambda_i}\lambda_i^n}{n!}.
\end{align*}
This is exactly the definition of $P^{(1)}(N = n)$.  Thus, the claim holds for $s = 0$.

\medskip
\textit{Inductive Step}: Assume the statement is true for $s = k$.  That is, $M^{(k)} - 1$
is equivalent to $N^{(k+1)}$ and $E[N^{(k)}]$ follows the claim.  We need to show the
statement is true for $s = k + 1$.  We must determine the distribution of $M^{(k+1)} - 1$.

We have,
\[
    P(M^{(k+1)} - 1 = n)
    = P(M^{(k+1)} = n + 1)
    = \frac{(n+1)\,P^{(k+1)}(N = n+1)}{E[N^{(k+1)}]}.
\]
First, we compute $E[N^{(k+1)}]$:
\begin{align*}
E[N^{(k+1)}]
    &= \sum_{n=0}^{\infty} n \cdot
       \sum_{i=1}^{t}
       \left(\frac{\beta_i\,\lambda_i^{k+1}}{\sum_{j=1}^{t}\beta_j\,\lambda_j^{k+1}}\right)
       \cdot \frac{e^{-\lambda_i}\lambda_i^n}{n!} \\
    &= \sum_{i=1}^{t}
       \left(\frac{\beta_i\,\lambda_i^{k+1}}{\sum_{j=1}^{t}\beta_j\,\lambda_j^{k+1}}\right)
       \cdot \lambda_i
     = \frac{\sum_{i=1}^{t}\beta_i\,\lambda_i^{k+2}}
            {\sum_{j=1}^{t}\beta_j\,\lambda_j^{k+1}}.
\end{align*}
This confirms Claim~1 for $s = k + 1$.  Next, we substitute this into the probability
expression:
\begin{align*}
P(M^{(k+1)} - 1 = n)
    &= \frac{n+1}
      {\displaystyle\frac{\sum_{i=1}^{t}\beta_i\lambda_i^{k+2}}
                         {\sum_{j=1}^{t}\beta_j\lambda_j^{k+1}}}
       \sum_{i=1}^{t}
       \left(\frac{\beta_i\lambda_i^{k+1}}{\sum_{j=1}^{t}\beta_j\lambda_j^{k+1}}\right)
       \cdot \frac{e^{-\lambda_i}\lambda_i^{n+1}}{(n+1)!}.
\end{align*}
We cancel the common denominator terms $\sum_{j=1}^{t}\beta_j\lambda_j^{k+1}$ and we have
\begin{align*}
    &= \frac{n+1}{\sum_{j=1}^{t}\beta_j\lambda_j^{k+2}}
       \sum_{i=1}^{t}\beta_i\lambda_i^{k+1}
       \cdot \frac{e^{-\lambda_i}\lambda_i^{n+1}}{(n+1)!} \\
    &= \sum_{i=1}^{t}
       \frac{\beta_i\lambda_i^{k+2}}{\sum_{j=1}^{t}\beta_j\lambda_j^{k+2}}
       \cdot \frac{e^{-\lambda_i}\lambda_i^n}{n!}.
\end{align*}
This expression is exactly the PMF for $N^{(k+2)}$ as desired.

So, by the principle of mathematical induction, for all $s \in \mathbb{N}$, the compound
identity variable associated with $N^{(s)}$ shifts the distribution to $N^{(s+1)}$, and
the expected value satisfies the derived closed form.
\end{proof}

Now we are ready to construct our new recursive expression by applying
Corollary~\ref{cor:main}.  From
\[
    P(S_N = k) = \frac{1}{k}\,E[N]\sum_{i=1}^{k} i\,\alpha_i\,P(S_{M-1} = k - i),
\]
we arrive at the updated recursive equation
\[
    P^{(s)}(S_N = k)
    = \frac{1}{k}
      \cdot \frac{\sum_{i=1}^{t}\beta_i\,\lambda_i^{s+1}}
                 {\sum_{i=1}^{t}\beta_i\,\lambda_i^s}
      \cdot \sum_{i=1}^{k} i\,\alpha_i \cdot P^{(s+1)}(S_N = k - i).
\]

\begin{example}[Revisiting Example~\ref{ex:poisson}]\label{ex:revisit-poisson}
Now we can compute the same compound random variable from
Example~\ref{ex:poisson}, but with our updated recursive equation.  Like before, let $N$ be
a random variable of the Poisson distribution with parameter $\lambda_1 = 3$.  So the
probability mass function for the distribution of $N$ is
\[
    P^{(s)}(N = n) = \beta_1 \cdot \frac{e^{-\lambda}\,\lambda_1^n}{n!},
    \quad \text{where } \beta_1 = 1.
\]
Furthermore,
\[
    E[N^{(s)}]
    = \frac{\sum_{i=1}^{1}\beta_i\,\lambda_i^{s+1}}
           {\sum_{i=1}^{1}\beta_i\,\lambda_i^s}
    = \frac{\beta_1\,\lambda_1^{s+1}}{\beta_1\,\lambda_1^s}
    = \frac{\lambda_1^{s+1}}{\lambda_1^s}
    = \lambda_1 = 3, \qquad \forall\, s \in \mathbb{Z},\; s \geq 0.
\]
Now we calculate $P^{(0)}(S_N = 4)$.  We want to find
\[
    P^{(0)}(S_N = 4)
    = \frac{3}{4}\bigl[0.05 \cdot P^{(1)}(S_N = 3) + 0.8 \cdot P^{(1)}(S_N = 2)
      + 0.3 \cdot P^{(1)}(S_N = 1) + 4(0.25) \cdot P^{(1)}(S_N = 0)\bigr].
\]
We can solve each of these recursive expressions for all $s \in \mathbb{Z}$, $s \geq 0$.
So we have,
\begin{align*}
P^{(s)}(S_N = 0) &= P(N^{(s)} = 0) = 1 \cdot \frac{e^{-\lambda}\,\lambda^0}{0!}
    = e^{-\lambda} = e^{-3}, \qquad \forall\, s \geq 0. \\
P^{(s)}(S_N = 1) &= \frac{3}{1}\bigl[1 \cdot (0.05) \cdot P(S_N = 0)\bigr]
    = 3(0.05)\,e^{-3} = 0.15\,e^{-3}, \qquad \forall\, s \geq 0. \\
P^{(s)}(S_N = 2) &= \frac{3}{2}\bigl[(0.05)\cdot P(S_N = 1) + 2(0.4)\cdot P(S_N = 0)\bigr] \\
    &= \frac{3}{2}\bigl[(0.05)(0.15\,e^{-3}) + (0.8)\,e^{-3}\bigr]
    = 1.21125\,e^{-3}, \qquad \forall\, s \geq 0. \\
P^{(s)}(S_N = 3) &= \frac{3}{3}\bigl[(0.05)\cdot P(S_N = 2) + (0.8)\cdot P(S_N = 1)
    + 3(0.1)\cdot P(S_N = 0)\bigr] \\
    &= (0.05)(1.21125\,e^{-3}) + (0.8)(0.15\,e^{-3}) + (0.3)\,e^{-3}
    = 0.4805625\,e^{-3}, \qquad \forall\, s \geq 0.
\end{align*}
Now we have everything we need to compute $P^{(0)}(S_N = 4)$.  We have,
\begin{align*}
P^{(0)}(S_N = 4)
    &= \frac{3}{4}\bigl[0.05\cdot P^{(1)}(S_N = 3) + 0.8\cdot P^{(1)}(S_N = 2)
       + 0.3\cdot P^{(1)}(S_N = 1) + 4(0.25)\cdot P^{(1)}(S_N = 0)\bigr] \\
    &= \frac{3}{4}\bigl[0.05(0.4805625\,e^{-3}) + 0.8(1.21125\,e^{-3})
       + 0.3(0.15\,e^{-3}) + e^{-3}\bigr] \\
    &= 1.528521094\,e^{-3}.
\end{align*}
Notice, this is the same as our result from Example~\ref{ex:poisson}.
\end{example}

\begin{example}[Revisiting Example~\ref{ex:poisson-mix}]\label{ex:revisit-mix}
We will consider again the case of a discrete mixture distribution where the event is
generated by one of two random processes.  The event is either generated by process one,
which follows a Poisson distribution with mean $\lambda_1$, or process two, which follows
a Poisson distribution with mean $\lambda_2$.  Like in Example~\ref{ex:poisson-mix}, we
let $\lambda_1 = 3$, $\lambda_2 = 4$, $\beta_1^{(0)} = 0.6$, $\beta_2^{(0)} = 0.4$.  We
want to find $P^{(0)}(S_N = 4)$.

So we have,
\[
    P^{(s)}(S_N = k)
    = \frac{1}{k}
      \left[
        \frac{(0.6)(3)^{s+1} + (0.4)(4)^{s+1}}
             {(0.6)(3)^s    + (0.4)(4)^s}
      \right]
      \sum_{i=1}^{k} i\,\alpha_i \cdot P^{(s+1)}(S_N = k - i).
\]
\begin{align*}
P^{(0)}(S_N = 4)
    &= \frac{1}{4}
       \left[\frac{(0.6)(3)^1 + (0.4)(4)^1}{(0.6)(3)^0 + (0.4)(4)^0}\right]
       \sum_{i=1}^{4} i\,\alpha_i \cdot P^{(1)}(S_N = 4 - i).
\end{align*}

\noindent\textit{Base cases}:
\begin{align*}
P^{(1)}(S_N = 0)
    &= \frac{(0.6)(3)^1}{(0.6)(3)^1 + (0.4)(4)^1}\,e^{-3}
     + \frac{(0.4)(4)^1}{(0.6)(3)^1 + (0.4)(4)^1}\,e^{-4}
     = \frac{1.8}{3.4}\,e^{-3} + \frac{1.6}{3.4}\,e^{-4}. \\
P^{(2)}(S_N = 0)
    &= \frac{(0.6)(3)^2}{(0.6)(3)^2 + (0.4)(4)^2}\,e^{-3}
     + \frac{(0.4)(4)^2}{(0.6)(3)^2 + (0.4)(4)^2}\,e^{-4}
     = \frac{5.4}{11.8}\,e^{-3} + \frac{6.4}{11.8}\,e^{-4}. \\
P^{(3)}(S_N = 0)
    &= \frac{16.2}{41.8}\,e^{-3} + \frac{25.6}{41.8}\,e^{-4}. \\
P^{(4)}(S_N = 0)
    &= \frac{48.6}{151}\,e^{-3} + \frac{102.4}{151}\,e^{-4}.
\end{align*}

\noindent\textit{Computing} $P^{(s)}(S_N = 1)$:
\begin{align*}
P^{(1)}(S_N = 1)
    &= \left[\frac{(0.6)(3)^2 + (0.4)(4)^2}{(0.6)(3)^1 + (0.4)(4)^1}\right]
       \bigl[1 \cdot 0.05 \cdot P^{(2)}(S_N = 0)\bigr] \\
    &= \frac{11.8}{3.4}\left[0.05\!\left(\frac{5.4}{11.8}\,e^{-3}
       + \frac{6.4}{11.8}\,e^{-4}\right)\right]
     = 0.07941176471\,e^{-3} + 0.09411764706\,e^{-4}. \\
P^{(2)}(S_N = 1)
    &= \frac{41.8}{11.8}\bigl[0.05 \cdot P^{(3)}(S_N = 0)\bigr]
     = 0.06864406780\,e^{-3} + 0.10847457627\,e^{-4}. \\
P^{(3)}(S_N = 1)
    &= \frac{151}{41.8}\bigl[0.05 \cdot P^{(4)}(S_N = 0)\bigr]
     = 0.05813397129\,e^{-3} + 0.12248803828\,e^{-4}.
\end{align*}

\noindent\textit{Computing} $P^{(s)}(S_N = 2)$:
\begin{align*}
P^{(1)}(S_N = 2)
    &= \frac{1}{2}\cdot\frac{11.8}{3.4}
       \bigl[(0.05)\cdot P^{(2)}(S_N = 1) + (0.8)\cdot P^{(2)}(S_N = 0)\bigr] \\
    &= 0.64125\,e^{-3} + 0.76235294118\,e^{-4}. \\
P^{(2)}(S_N = 2)
    &= \frac{1}{2}\cdot\frac{41.8}{11.8}
       \bigl[(0.05)\cdot P^{(3)}(S_N = 1) + (0.8)\cdot P^{(3)}(S_N = 0)\bigr] \\
    &= 0.55430084746\,e^{-3} + 0.87864406780\,e^{-4}.
\end{align*}

\noindent\textit{Computing} $P^{(1)}(S_N = 3)$:
\begin{align*}
P^{(1)}(S_N = 3)
    &= \frac{1}{3}\cdot\frac{11.8}{3.4}
       \bigl[(0.05)\cdot P^{(2)}(S_N = 2) + (0.8)\cdot P^{(2)}(S_N = 1)
       + (0.3)\cdot P^{(2)}(S_N = 0)\bigr] \\
    &= 0.25441544118\,e^{-3} + 0.33945098039\,e^{-4}.
\end{align*}

\noindent\textit{Finally,} $P^{(0)}(S_N = 4)$:
\begin{align*}
P^{(0)}(S_N = 4)
    &= \frac{1}{4}\cdot\frac{3.4}{1}
       \bigl[(0.05)\cdot P^{(1)}(S_N = 3) + (0.8)\cdot P^{(1)}(S_N = 2) \\
    &\qquad\qquad\quad + (0.3)\cdot P^{(1)}(S_N = 1)
       + P^{(1)}(S_N = 0)\bigr] \\
    &= \boxed{0.91711265625\,e^{-3} + 0.95682666667\,e^{-4}}.
\end{align*}
\end{example}

%% ============================================================
\section{Generalized Computation with a Finite Mixture of Distributions}
%% ============================================================

We have shown in Example~\ref{ex:poisson} with $N$ in the Poisson distribution that
\[
    P^{(s+1)}(M^{(s)} - 1 = n) = P^{(s)}(N = n).
\]
We have shown in Example~\ref{ex:negbin} with $N$ in the negative binomial distribution that
\[
    P^{(s)}(M^{(s)} - 1 = n) = P^{(s+1)}(N = n),
\]
where the negative binomial parameter shifts from $r + s$ to $r + s + 1$.  Similarly, we
have shown in Example~\ref{ex:binom} with $N$ in the binomial distribution that
\[
    P^{(s)}(M^{(s)} - 1 = n) = P^{(s+1)}(N = n),
\]
where the binomial parameter shifts from $r - s$ to $r - s - 1$.

Notice that for each of these distributions, when calculating $P(M^{(s)} - 1 = n)$, at a
given level of recursion, we arrive at either $P^{(s)}(N = n)$ or a version of
$P^{(s)}(N = n)$ with a parameter value change that is a function of the level of
recursion.

We will now show that by moving the level of recursion to a parameter of the probability
mass function of the finite mixture of distributions we can compute a compound random
variable with the counting distribution in any finite mixture of any of these three
distributions.

\begin{corollary}[Recursive Properties of the Finite Mixture Distributions]
\label{cor:general}
Let $s \in \mathbb{Z}$ with $0 \leq s$, and let $s$ represent the level of recursion.
Let $N^{(s)}$ be a mixed random variable of Binomial or Negative Binomial distribution at
recursion level $s$ with probability mass function:
\[
    P^{(s)}(N = n) = \sum_{j=1}^{t} \beta_j^{(s)}\, P_{r_j}(N^{(s)} = n), \qquad n \geq 0,
\]
where $\sum_{i=1}^{t}\beta_i = 1$.  Let $M^{(s)}$ be the random variable satisfying the
compound identity condition associated with $N^{(s)}$.  We claim that for all $s \geq 0$:
\begin{enumerate}
\item $E[N^{(s)}] = \displaystyle\sum_{j=1}^{t} \beta_j^{(s)}\, E[N_j^{(0)}].$
\item The distribution of $M^{(s)} - 1$ is identical to the distribution of $N^{(s+1)}$,
      where the new weights are updated such that
      \[
          \beta_j^{(s+1)} = \beta_j^{(s)} \cdot
          \frac{E[N_j^{(s)}]}{E[N^{(s)}]}.
      \]
\end{enumerate}
\end{corollary}

\begin{proof}[Proof (by induction)]
\textit{Base Case} $(s = 0)$: For $s = 0$, the weights are $\beta_i$.  First, we compute
the expected value of $E[N^{(0)}]$:
\begin{align*}
E[N^{(0)}]
    &= \sum_{n=0}^{\infty} n \cdot \left[\sum_{j=1}^{t}\beta_j^{(0)}\,P_{r_j}(n)\right]
     = \sum_{j=1}^{t}\beta_j^{(0)}\,E[N_j^{(0)}].
\end{align*}
Thus, Claim~1 holds for case $s = 0$.

Next, we apply the compound random identity and see that,
\begin{align*}
P(M^{(0)} - 1 = n)
    &= P(M^{(0)} = n+1)
     = \frac{(n+1)\,P(N^{(0)} = n+1)}{E[N^{(0)}]} \\
    &= \frac{(n+1)\!\left[\sum_{j=1}^{t}\beta_j^{(0)}\,P_{r_j}(N^{(0)} = n+1)\right]}
      {E[N^{(0)}]} \\
    &= \frac{1}{E[N^{(0)}]}
       \sum_{j=1}^{t}\beta_j^{(0)}\!\left[(n+1)\cdot P_{r_j}(N^{(0)} = n+1)\right].
\end{align*}
Notice that for the negative binomial distribution,
$P^{(s)}(M^{(s)} - 1 = n) = P^{(s+1)}(N = n)$ implies
\[
    (n+1)\,P_r(N = n+1) = E[N]\, P_{r+1}(N = n).
\]
So by substitution we have,
\begin{align*}
P(M^{(0)} - 1 = n)
    &= \frac{1}{E[N^{(0)}]}
       \sum_{j=1}^{t}\beta_j^{(0)}\, E[N_j^{(0)}] \cdot P_{r_j+1}(N^{(1)} = n) \\
    &= \sum_{j=1}^{t}
       \frac{\beta_j\, E[N_j^{(0)}]}{E[N^{(0)}]} \cdot P_{r_j+1}(N^{(1)} = n).
\end{align*}
Similarly, for the binomial distribution,
$P^{(s)}(M^{(s)} - 1 = n) = P^{(s+1)}(N = n)$ implies
\[
    (n+1)\,P_r(N = n+1) = E[N]\, P_{r-1}(N = n).
\]
So by substitution we have,
\begin{align*}
P(M^{(0)} - 1 = n)
    &= \frac{1}{E[N^{(0)}]}
       \sum_{j=1}^{t}\beta_j^{(0)}\, E[N_j^{(0)}] \cdot P_{r_j-1}(N^{(1)} = n) \\
    &= \sum_{j=1}^{t}
       \frac{\beta_j\, E[N_j^{(0)}]}{E[N^{(0)}]} \cdot P_{r_j-1}(N^{(1)} = n).
\end{align*}
These match the definition of $N^{(1)}$, a mixture with shifted parameters according to
the distribution.  Thus, the claim holds for $s = 0$.

\medskip
\textit{Inductive Step}: Assume the statement is true for $s = k$.  That is, $M^{(k)} - 1$
is equivalent to $N^{(k+1)}$ and $E[N^{(k)}]$ follows the claim.  We need to show the
statement is true for $s = k + 1$.  We must determine the distribution of $M^{(k+1)} - 1$.

We have,
\begin{align*}
P(M^{(k+1)} - 1 = n)
    &= P(M^{(k+1)} = n+1)
     = \frac{(n+1)\,P(N^{(k+1)} = n+1)}{E[N^{(k+1)}]}.
\end{align*}
Substituting the mixture sum for $N^{(k+1)}$,
\begin{align*}
P(M^{(k+1)} - 1 = n)
    &= \frac{1}{E[N^{(k+1)}]}
       \sum_{j=1}^{t}\beta_j^{(k+1)}
       \!\left[(n+1)\cdot P_{r_j+(k+1)}(N^{(0)} = n+1)\right].
\end{align*}
So, for the negative binomial distribution, by substitution we have,
\begin{align*}
P(M^{(k+1)} - 1 = n)
    &= \frac{1}{E[N^{(k+1)}]}
       \sum_{j=1}^{t}\beta_j^{(k+1)}\, E[N_j^{(k+1)}]
       \cdot P_{r_j+k+2}(N^{(1)} = n) \\
    &= \sum_{j=1}^{t}
       \frac{\beta_j^{k+1}\, E[N_j^{(k+1)}]}{E[N^{(k+1)}]}
       \cdot P_{r_j+k+2}(N^{(1)} = n).
\end{align*}
Similarly, for the binomial distribution, by substitution we have,
\begin{align*}
P(M^{(k+1)} - 1 = n)
    &= \frac{1}{E[N^{(k+1)}]}
       \sum_{j=1}^{t}\beta_j^{(k+1)}\, E[N_j^{(k+1)}]
       \cdot P_{r_j-(k+2)}(N^{(1)} = n) \\
    &= \sum_{j=1}^{t}
       \frac{\beta_j^{k+1}\, E[N_j^{(k+1)}]}{E[N^{(k+1)}]}
       \cdot P_{r_j-(k+2)}(N^{(1)} = n).
\end{align*}
This expression is exactly the PMF for $N^{(k+2)}$ with weights $\beta_j^{k+1}$ as desired.

So, by the principle of mathematical induction, for all $s \in \mathbb{N}$, the compound
identity variable associated with $N^{(s)}$ shifts the distribution to $N^{(s+1)}$, and
the expected value satisfies the derived closed form.
\end{proof}

Now we are ready to construct our new recursive expression by applying
Corollary~\ref{cor:main}.  From
\[
    P(S_N = k) = \frac{1}{k}\,E[N]\sum_{i=1}^{k} i\,\alpha_i\,P(S_{M-1} = k - i),
\]
we arrive at the updated recursive equation
\[
    P^{(s)}(S_N = k)
    = \frac{E[N^{(s)}]}{k} \cdot \sum_{i=1}^{k} i\,\alpha_i \cdot P^{(s+1)}(S_N = k - i),
\]
where $E[N^{(s)}] = \sum_{j=1}^{t}\beta_j^{(s)}\,E[N_j^{(0)}]$ and the computation of
$P^{(s+1)}$ uses $\beta^{(s+1)}$ and parameters appropriate to the distribution of $P_i$,
that is, $(r - s)$ for the binomial distribution, $(r + s)$ for the negative binomial
distribution and no parameter change for the Poisson distribution.

\begin{example}[N in a Finite Mixture of Binomial Distributions]\label{ex:binom-mix}
Consider a discrete mixture distribution where the event is generated by one of two random
processes.  The event is either generated by process one, which follows a Binomial
distribution with $r_1 = 8$ and $p_1 = 0.5$, or process two, which follows a Binomial
distribution with $r_2 = 10$ and $p_2 = 0.7$.  The probability of process one is
$\beta_1^{(0)} = 0.6$.  The probability of process two is $\beta_2^{(0)} = 0.4$.  Notice
that $\beta_1^{(0)} + \beta_2^{(0)} = 1$.  We want to find $P(S_N = 4)$.

So, the probability mass function for this mixed distribution is
\[
    P^{(s)}(N = n)
    = \sum_{j=1}^{2}\beta_j^{(s)}
      \binom{r_j - s}{n} p_j^n(1-p_j)^{(r_j - s) - n},
    \quad 0 \leq n \leq \max(r_1 - s,\, r_2 - s).
\]

\noindent\textit{Computing the mixing weights at each level}:
\begin{align*}
E[N^{(0)}]
    &= \beta_1^{(0)}(r_1 - 0)p_1 + \beta_2^{(0)}(r_2 - 0)p_2
     = 0.6 \cdot 8 \cdot 0.5 + 0.4 \cdot 10 \cdot 0.7
     = 5.2. \\
\beta_1^{(1)} &= \beta_1^{(0)} \cdot \frac{E[N_1^{(0)}]}{E[N^{(0)}]}
    = \frac{2.4}{5.2} = 0.46153846154. \\
\beta_2^{(1)} &= \beta_2^{(0)} \cdot \frac{E[N_2^{(0)}]}{E[N^{(0)}]}
    = \frac{2.8}{5.2} = 0.53846153846. \\
E[N^{(1)}]
    &= \beta_1^{(1)}(r_1 - 1)p_1 + \beta_2^{(1)}(r_2 - 1)p_2 \\
    &= 0.46153846154 \cdot 7 \cdot 0.5 + 0.53846153846 \cdot 9 \cdot 0.7
     = 5.00769230769. \\
\beta_1^{(2)} &= \beta_1^{(1)} \cdot \frac{E[N_1^{(1)}]}{E[N^{(1)}]}
    = \frac{1.61538461538}{5.00769230769} = 0.32258064516. \\
\beta_2^{(2)} &= \beta_2^{(1)} \cdot \frac{E[N_2^{(1)}]}{E[N^{(1)}]}
    = \frac{3.39230769231}{5.00769230769} = 0.67741935484. \\
E[N^{(2)}]
    &= \beta_1^{(2)}(r_1 - 2)p_1 + \beta_2^{(2)}(r_2 - 2)p_2 \\
    &= 0.32258064516 \cdot 6 \cdot 0.5 + 0.67741935484 \cdot 8 \cdot 0.7
     = 4.76129032258. \\
\beta_1^{(3)} &= \beta_1^{(2)} \cdot \frac{E[N_1^{(2)}]}{E[N^{(2)}]}
    = \frac{0.96774193548}{4.76129032258} = 0.20325203252. \\
\beta_2^{(3)} &= \beta_2^{(2)} \cdot \frac{E[N_2^{(2)}]}{E[N^{(2)}]}
    = \frac{3.79354838710}{4.76129032258} = 0.79674796748. \\
E[N^{(3)}]
    &= \beta_1^{(3)}(r_1 - 3)p_1 + \beta_2^{(3)}(r_2 - 3)p_2 \\
    &= 0.20325203252 \cdot 5 \cdot 0.5 + 0.79674796748 \cdot 7 \cdot 0.7
     = 4.41219512195. \\
\beta_1^{(4)} &= \beta_1^{(3)} \cdot \frac{E[N_1^{(3)}]}{E[N^{(3)}]}
    = \frac{0.50813008130}{4.41219512195} = 0.11516491616. \\
\beta_2^{(4)} &= \beta_2^{(3)} \cdot \frac{E[N_2^{(3)}]}{E[N^{(3)}]}
    = \frac{3.90406504065}{4.41219512195} = 0.88483508384. \\
E[N^{(4)}]
    &= \beta_1^{(4)}(r_1 - 4)p_1 + \beta_2^{(4)}(r_2 - 4)p_2 \\
    &= 0.11516491616 \cdot 4 \cdot 0.5 + 0.88483508384 \cdot 6 \cdot 0.7
     = 3.94663718445.
\end{align*}

Now we can begin the computation to find $P^{(0)}(S_N = 4)$.  We want to find,
\[
    P^{(0)}(S_N = 4)
    = \frac{E[N^{(0)}]}{4}
      \bigl[0.05 \cdot P^{(1)}(S_N = 3) + 0.8 \cdot P^{(1)}(S_N = 2)
      + 0.3 \cdot P^{(1)}(S_N = 1) + P^{(1)}(S_N = 0)\bigr].
\]

\noindent\textit{Base cases}:
\[
    P^{(s)}(S_N = 0)
    = \beta_1^{(s)} \cdot (1-p_1)^{r_1-s} + \beta_2^{(s)} \cdot (1-p_2)^{r_2-s}
    = \beta_1^{(s)} \cdot (0.5)^{8-s} + \beta_2^{(s)} \cdot (0.3)^{10-s}.
\]
\begin{align*}
P^{(1)}(S_N = 0)
    &= 0.46153846154\,(0.5)^7 + 0.53846153846\,(0.3)^9
     = 0.00361636777. \\
P^{(2)}(S_N = 0)
    &= 0.32258064516\,(0.5)^6 + 0.67741935484\,(0.3)^8
     = 0.00508476806. \\
P^{(3)}(S_N = 0)
    &= 0.20325203252\,(0.5)^5 + 0.79674796748\,(0.3)^7
     = 0.00652587480. \\
P^{(4)}(S_N = 0)
    &= 0.11516491616\,(0.5)^4 + 0.88483508384\,(0.3)^6
     = 0.00784285204.
\end{align*}

\noindent\textit{Computing} $P^{(s)}(S_N = 1)$:
\begin{align*}
P^{(1)}(S_N = 1)
    &= \frac{E[N^{(1)}]}{1}\bigl[0.05 \cdot P^{(2)}(S_N = 0)\bigr]
     = 5.00769230769(0.05)(0.00508476806) = 0.00127314770. \\
P^{(2)}(S_N = 1)
    &= \frac{E[N^{(2)}]}{1}\bigl[0.05 \cdot P^{(3)}(S_N = 0)\bigr]
     = 4.76129032258(0.05)(0.00652587480) = 0.00155357923. \\
P^{(3)}(S_N = 1)
    &= \frac{E[N^{(3)}]}{1}\bigl[0.05 \cdot P^{(4)}(S_N = 0)\bigr]
     = 4.41219512195(0.05)(0.00784285204) = 0.00173020967.
\end{align*}

\noindent\textit{Computing} $P^{(s)}(S_N = 2)$:
\begin{align*}
P^{(1)}(S_N = 2)
    &= \frac{E[N^{(1)}]}{2}
       \bigl[0.05 \cdot P^{(2)}(S_N = 1) + 0.8 \cdot P^{(2)}(S_N = 0)\bigr] \\
    &= \frac{5.00769230769}{2}
       \bigl[0.05(0.00155357923) + 0.8(0.00508476806)\bigr]
     = 0.01037967774. \\
P^{(2)}(S_N = 2)
    &= \frac{E[N^{(2)}]}{2}
       \bigl[0.05 \cdot P^{(3)}(S_N = 1) + 0.8 \cdot P^{(3)}(S_N = 0)\bigr] \\
    &= \frac{4.76129032258}{2}
       \bigl[0.05(0.00173020967) + 0.8(0.00652587480)\bigr]
     = 0.01263458457.
\end{align*}

\noindent\textit{Computing} $P^{(1)}(S_N = 3)$:
\begin{align*}
P^{(1)}(S_N = 3)
    &= \frac{E[N^{(1)}]}{3}
       \bigl[0.05 \cdot P^{(2)}(S_N = 2) + 0.8 \cdot P^{(2)}(S_N = 1)
       + 0.3 \cdot P^{(2)}(S_N = 0)\bigr] \\
    &= \frac{5.00769230769}{3}
       \bigl[0.05(0.01263458457) + 0.8(0.00155357923)
       + 0.3(0.00508476806)\bigr] \\
    &= 0.00567542306.
\end{align*}

\noindent\textit{Computing} $P^{(0)}(S_N = 4)$:
\begin{align*}
P^{(0)}(S_N = 4)
    &= \frac{E[N^{(0)}]}{4}
       \bigl[0.05 \cdot P^{(1)}(S_N = 3) + 0.8 \cdot P^{(1)}(S_N = 2)
       + 0.3 \cdot P^{(1)}(S_N = 1) + P^{(1)}(S_N = 0)\bigr] \\
    &= \frac{5.2}{4}
       \bigl[0.05(0.00567542306) + 0.8(0.01037967774) \\
    &\qquad\qquad + 0.3(0.00127314770) + 0.00361636777\bigr] \\
    &= \boxed{0.01636157305}.
\end{align*}
\end{example}

\begin{example}[N in a Finite Mixture of Negative Binomial Distributions]
\label{ex:negbin-mix}
Consider a discrete mixture distribution where the event is generated by one of two random
processes.  The event is either generated by process one, which follows a Negative Binomial
distribution with $r_1 = 4$ and $p_1 = 0.6$, or process two, which follows a Negative
Binomial distribution with $r_2 = 6$ and $p_2 = 0.5$.  The probability of process one is
$\beta_1^{(0)} = 0.5$.  The probability of process two is $\beta_2^{(0)} = 0.5$.  Notice
that $\beta_1^{(0)} + \beta_2^{(0)} = 1$.  We want to find $P(S_N = 4)$.

So, the probability mass function for this mixed distribution is
\[
    P^{(s)}(N = n)
    = \sum_{j=1}^{2}\beta_j^{(s)}
      \binom{n + r_j + s - 1}{n} p_j^{r_j+s}(1-p_j)^n,
    \quad 0 \leq n \leq \max(r_1 + s,\, r_2 + s).
\]

\noindent\textit{Computing the mixing weights at each level}:
\begin{align*}
E[N^{(0)}]
    &= 0.5\left(\frac{(4+0)(0.4)}{0.6}\right) + 0.5\left(\frac{(6+0)(0.5)}{0.5}\right)
     = 4.33333333333. \\
\beta_1^{(1)} &= 0.5 \cdot \frac{1.33333333333}{4.33333333333} = 0.30769230769. \\
\beta_2^{(1)} &= 0.5 \cdot \frac{3}{4.33333333333} = 0.69230769231. \\
E[N^{(1)}]
    &= 0.30769230769 \cdot \frac{5 \cdot 0.4}{0.6}
     + 0.69230769231 \cdot \frac{7 \cdot 0.5}{0.5}
     = 5.87179487179. \\
\beta_1^{(2)} &= 0.30769230769 \cdot
    \frac{1.02564102564}{5.87179487179} = 0.17467248908. \\
\beta_2^{(2)} &= 0.69230769231 \cdot
    \frac{4.84615384615}{5.87179487179} = 0.82532751092. \\
E[N^{(2)}]
    &= 0.17467248908 \cdot \frac{6 \cdot 0.4}{0.6}
     + 0.82532751092 \cdot \frac{8 \cdot 0.5}{0.5}
     = 7.30131004367. \\
\beta_1^{(3)} &= 0.17467248908 \cdot
    \frac{0.69868995633}{7.30131004367} = 0.09569377990. \\
\beta_2^{(3)} &= 0.82532751092 \cdot
    \frac{6.60262008734}{7.30131004367} = 0.90430622010. \\
E[N^{(3)}]
    &= 0.09569377990 \cdot \frac{7 \cdot 0.4}{0.6}
     + 0.90430622010 \cdot \frac{9 \cdot 0.5}{0.5}
     = 8.58532695375. \\
\beta_1^{(4)} &= 0.09569377990 \cdot
    \frac{0.44664430295}{8.58532695375} = 0.05201560468. \\
\beta_2^{(4)} &= 0.90430622010 \cdot
    \frac{8.13875598086}{8.58532695375} = 0.94798439532. \\
E[N^{(4)}]
    &= 0.05201560468 \cdot \frac{8 \cdot 0.4}{0.6}
     + 0.94798439532 \cdot \frac{10 \cdot 0.5}{0.5}
     = 9.75726051149.
\end{align*}

Now we can begin the computation to find $P^{(0)}(S_N = 4)$.  We want to find,
\[
    P^{(0)}(S_N = 4)
    = \frac{E[N^{(0)}]}{4}
      \bigl[0.05 \cdot P^{(1)}(S_N = 3) + 0.8 \cdot P^{(1)}(S_N = 2)
      + 0.3 \cdot P^{(1)}(S_N = 1) + P^{(1)}(S_N = 0)\bigr].
\]

\noindent\textit{Base cases}
$P^{(s)}(S_N = 0) = \beta_1^{(s)}\,p_1^{r_1+s} + \beta_2^{(s)}\,p_2^{r_2+s}$:
\begin{align*}
P^{(0)}(S_N = 0)
    &= 0.5(0.6)^4 + 0.5(0.5)^6 = 0.0726125. \\
P^{(1)}(S_N = 0)
    &= 0.30769230769(0.6)^5 + 0.69230769231(0.5)^7
     = 0.02933480769. \\
P^{(2)}(S_N = 0)
    &= 0.17467248908(0.6)^6 + 0.82532751092(0.5)^8
     = 0.01137345524. \\
P^{(3)}(S_N = 0)
    &= 0.09569377990(0.6)^7 + 0.90430622010(0.5)^9
     = 0.00444503648. \\
P^{(4)}(S_N = 0)
    &= 0.05201560468(0.6)^8 + 0.94798439532(0.5)^{10}
     = 0.00179942843.
\end{align*}

\noindent\textit{Computing} $P^{(s)}(S_N = 1)$:
\begin{align*}
P^{(1)}(S_N = 1)
    &= \frac{E[N^{(1)}]}{1}\bigl[0.05 \cdot P^{(2)}(S_N = 0)\bigr]
     = 5.87179487179(0.05)(0.01137345524) = 0.00333912981. \\
P^{(2)}(S_N = 1)
    &= \frac{E[N^{(2)}]}{1}\bigl[0.05 \cdot P^{(3)}(S_N = 0)\bigr]
     = 7.30131004367(0.05)(0.00444503648) = 0.00162272948. \\
P^{(3)}(S_N = 1)
    &= \frac{E[N^{(3)}]}{1}\bigl[0.05 \cdot P^{(2)}(S_N = 0)\bigr]
     = 8.58532695375(0.05)(0.01137345524) = 0.00488224159.
\end{align*}

\noindent\textit{Computing} $P^{(s)}(S_N = 2)$:
\begin{align*}
P^{(1)}(S_N = 2)
    &= \frac{E[N^{(1)}]}{2}
       \bigl[0.05 \cdot P^{(2)}(S_N = 1) + 0.8 \cdot P^{(2)}(S_N = 0)\bigr] \\
    &= \frac{5.87179487179}{2}
       \bigl[0.05(0.00162272948) + 0.8(0.01137345524)\bigr]
     = 0.02695124683. \\
P^{(2)}(S_N = 2)
    &= \frac{E[N^{(2)}]}{2}
       \bigl[0.05 \cdot P^{(3)}(S_N = 1) + 0.8 \cdot P^{(3)}(S_N = 0)\bigr] \\
    &= \frac{7.30131004367}{2}
       \bigl[0.05(0.00488224159) + 0.8(0.00444503648)\bigr]
     = 0.01387300480.
\end{align*}

\noindent\textit{Computing} $P^{(1)}(S_N = 3)$:
\begin{align*}
P^{(1)}(S_N = 3)
    &= \frac{E[N^{(1)}]}{3}
       \bigl[0.05 \cdot P^{(2)}(S_N = 2) + 0.8 \cdot P^{(2)}(S_N = 1)
       + 0.3 \cdot P^{(2)}(S_N = 0)\bigr] \\
    &= \frac{5.87179487179}{3}
       \bigl[0.05(0.01387300480) + 0.8(0.00162272948)
       + 0.3(0.01137345524)\bigr] \\
    &= 0.01057680615.
\end{align*}

\noindent\textit{Computing} $P^{(0)}(S_N = 4)$:
\begin{align*}
P^{(0)}(S_N = 4)
    &= \frac{E[N^{(0)}]}{4}
       \bigl[0.05 \cdot P^{(1)}(S_N = 3) + 0.8 \cdot P^{(1)}(S_N = 2)
       + 0.3 \cdot P^{(1)}(S_N = 1) + P^{(1)}(S_N = 0)\bigr] \\
    &= \frac{4.33333333333}{4}
       \bigl[0.05(0.01057680615) + 0.8(0.02695124683) \\
    &\qquad\qquad + 0.3(0.00333912981) + 0.02933480769\bigr] \\
    &= \boxed{0.05679524977}.
\end{align*}
\end{example}

\begin{example}[N in a Finite Mixture of Poisson, Binomial and Negative Binomial
Distributions]\label{ex:full-mix}
Consider a discrete mixture distribution where the event is generated by one of three
random processes.  The event is either generated by process one, which follows a Poisson
distribution with $\lambda = 3$, or process two, which follows a Binomial distribution with
$r_2 = 6$ and $p_2 = 0.5$, or process three, which follows a Negative Binomial
distribution with $r_3 = 2$ and $p_3 = 0.4$.  The probability of process one is
$\beta_1^{(0)} = 0.4$.  The probability of process two is $\beta_2^{(0)} = 0.3$.  The
probability of process three is $\beta_3^{(0)} = 0.3$.  Notice that
$\beta_1^{(0)} + \beta_2^{(0)} + \beta_3^{(0)} = 1$.  We want to find $P(S_N = 4)$.

So, the probability mass function for this mixed distribution is
\[
    P^{(s)}(N = n)
    = \beta_1^{(s)} \cdot \frac{e^{-\lambda}\lambda^n}{n!}
    + \beta_2^{(s)} \cdot \binom{r_2 - s}{n} p_2^n(1-p_2)^{(r_2-s)-n}
    + \beta_3^{(s)} \cdot \binom{n + r_3 + s - 1}{n} p_3^{r_3+s}(1-p_3)^n,
\]
for $0 \leq n \leq \max(r_2 - s,\, r_3 + s)$.

So we have,
\begin{align*}
E[N_1^{(s)}] &= \lambda = 3, \quad \text{for all } s. \\
E[N_2^{(s)}] &= (r_2 - s)\,p_2 = (6 - s)(0.5). \\
E[N_3^{(s)}] &= \frac{(r_3 + s)(1-p_3)}{p_3}
    = \frac{(2 + s)(0.6)}{0.4} = (2 + s)(1.5).
\end{align*}

\noindent\textit{Computing the mixing weights at each level}:
\begin{align*}
\beta_1^{(0)} &= 0.4, \quad \beta_2^{(0)} = 0.3, \quad \beta_3^{(0)} = 0.3. \\
E[N^{(0)}]
    &= (0.4)(3) + (0.3)(3) + (0.3)(3) = 3. \\
\beta_1^{(1)} &= 0.4 \cdot \frac{3}{3} = 0.4. \\
\beta_2^{(1)} &= 0.3 \cdot \frac{3}{3} = 0.3. \\
\beta_3^{(1)} &= 0.3 \cdot \frac{3}{3} = 0.3. \\
E[N^{(1)}]
    &= 0.4(3) + 0.3(2.5) + 0.3(4.5) = 3.3. \\
\beta_1^{(2)} &= 0.4 \cdot \frac{3}{3.3} = 0.36363636364. \\
\beta_2^{(2)} &= 0.3 \cdot \frac{2.5}{3.3} = 0.22727272727. \\
\beta_3^{(2)} &= 0.3 \cdot \frac{4.5}{3.3} = 0.40909090909. \\
E[N^{(2)}]
    &= 0.36363636364(3) + 0.22727272727(2) + 0.40909090909(6) = 4. \\
\beta_1^{(3)} &= 0.36363636364 \cdot \frac{3}{4} = 0.27272727273. \\
\beta_2^{(3)} &= 0.22727272727 \cdot \frac{2}{4} = 0.11363636364. \\
\beta_3^{(3)} &= 0.40909090909 \cdot \frac{6}{4} = 0.61363636364. \\
E[N^{(3)}]
    &= 0.27272727273(3) + 0.11363636364(1.5) \\
    &\quad + 0.61363636364(7.5) = 5.59090909091. \\
\beta_1^{(4)} &= 0.27272727273 \cdot \frac{3}{5.59090909091} = 0.14634146341. \\
\beta_2^{(4)} &= 0.11363636364 \cdot \frac{1.5}{5.59090909091} = 0.03048780488. \\
\beta_3^{(4)} &= 0.61363636364 \cdot
    \frac{7.5}{5.59090909091} = 0.82317073171. \\
E[N^{(4)}]
    &= 0.14634146341(3) + 0.03048780488(1) \\
    &\quad + 0.82317073171(9) = 7.87804878049.
\end{align*}

Now we can begin the computation to find $P^{(0)}(S_N = 4)$.  We want to find,
\[
    P^{(0)}(S_N = 4)
    = \frac{E[N^{(0)}]}{4}
      \bigl[0.05 \cdot P^{(1)}(S_N = 3) + 0.8 \cdot P^{(1)}(S_N = 2)
      + 0.3 \cdot P^{(1)}(S_N = 1) + P^{(1)}(S_N = 0)\bigr].
\]

\noindent\textit{Base cases}:
\[
    P^{(s)}(N = 0)
    = \beta_1^{(s)}\,e^{-3} + \beta_2^{(s)}\,(0.5)^{6-s}
    + \beta_3^{(s)}\,(0.4)^{2+s}.
\]
\begin{align*}
P^{(0)}(S_N = 0)
    &= 0.4\,e^{-3} + 0.3(0.5)^6 + 0.3(0.4)^2 = 0.07260232735. \\
P^{(1)}(S_N = 0)
    &= 0.4\,e^{-3} + 0.3(0.5)^5 + 0.3(0.4)^3 = 0.04848982735. \\
P^{(2)}(S_N = 0)
    &= 0.36363636364\,e^{-3} + 0.22727272727(0.5)^4
     + 0.40909090909(0.4)^4 = 0.04278166122. \\
P^{(3)}(S_N = 0)
    &= 0.27272727273\,e^{-3} + 0.11363636364(0.5)^3
     + 0.61363636364(0.4)^5 = 0.03406647319. \\
P^{(4)}(S_N = 0)
    &= 0.14634146341\,e^{-3} + 0.03048780488(0.5)^2
     + 0.82317073171(0.4)^6 = 0.01827957098.
\end{align*}

\noindent\textit{Computing} $P^{(s)}(S_N = 1)$:
\begin{align*}
P^{(1)}(S_N = 1)
    &= \frac{E[N^{(1)}]}{1}\bigl[0.05 \cdot P^{(2)}(S_N = 0)\bigr]
     = 3.3(0.05)(0.04278166122) = 0.00705897410. \\
P^{(2)}(S_N = 1)
    &= \frac{E[N^{(2)}]}{1}\bigl[0.05 \cdot P^{(3)}(S_N = 0)\bigr]
     = 4(0.05)(0.03406647319) = 0.00681329464. \\
P^{(3)}(S_N = 1)
    &= \frac{E[N^{(3)}]}{1}\bigl[0.05 \cdot P^{(4)}(S_N = 0)\bigr]
     = 5.59090909091(0.05)(0.01827957098) = 0.00510997098.
\end{align*}

\noindent\textit{Computing} $P^{(s)}(S_N = 2)$:
\begin{align*}
P^{(1)}(S_N = 2)
    &= \frac{E[N^{(1)}]}{2}
       \bigl[0.05 \cdot P^{(2)}(S_N = 1) + 0.8 \cdot P^{(2)}(S_N = 0)\bigr] \\
    &= \frac{3.3}{2}
       \bigl[0.05(0.00681329464) + 0.8(0.04278166122)\bigr]
     = 0.05703388962. \\
P^{(2)}(S_N = 2)
    &= \frac{E[N^{(2)}]}{2}
       \bigl[0.05 \cdot P^{(3)}(S_N = 1) + 0.8 \cdot P^{(3)}(S_N = 0)\bigr] \\
    &= \frac{4}{2}
       \bigl[0.05(0.00510997098) + 0.8(0.03406647319)\bigr]
     = 0.05501735420.
\end{align*}

\noindent\textit{Computing} $P^{(1)}(S_N = 3)$:
\begin{align*}
P^{(1)}(S_N = 3)
    &= \frac{E[N^{(1)}]}{3}
       \bigl[0.05 \cdot P^{(2)}(S_N = 2) + 0.8 \cdot P^{(2)}(S_N = 1)
       + 0.3 \cdot P^{(2)}(S_N = 0)\bigr] \\
    &= \frac{3.3}{3}
       \bigl[0.05(0.05501735420) + 0.8(0.00681329464)
       + 0.3(0.04278166122)\bigr] \\
    &= 0.02313960197.
\end{align*}

\noindent\textit{Computing} $P^{(0)}(S_N = 4)$:
\begin{align*}
P^{(0)}(S_N = 4)
    &= \frac{E[N^{(0)}]}{4}
       \bigl[0.05 \cdot P^{(1)}(S_N = 3) + 0.8 \cdot P^{(1)}(S_N = 2)
       + 0.3 \cdot P^{(1)}(S_N = 1) + P^{(1)}(S_N = 0)\bigr] \\
    &= \frac{3}{4}
       \bigl[0.05(0.02313960197) + 0.8(0.05703388962)
       + 0.3(0.00705897410) + 0.04848982735\bigr] \\
    &= \boxed{0.07304370853}.
\end{align*}
\end{example}

%% ============================================================
%  Bibliography (placeholder — update with actual references)
%% ============================================================
\begin{thebibliography}{9}

\bibitem{todo}
[TODO: Need the right list here. Potentials are Feller (1968), Grandell (1997),
and Johnson, Kotz, and Kemp (1992).]

\end{thebibliography}

\end{document}
